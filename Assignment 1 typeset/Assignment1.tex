\documentclass[11pt, oneside]{article}   	% use "amsart" instead of "article" for AMSLaTeX format
\usepackage{geometry}                		% See geometry.pdf to learn the layout options. There are lots.
\geometry{letterpaper}                   		% ... or a4paper or a5paper or ... 
%\geometry{landscape}                		% Activate for rotated page geometry
%\usepackage[parfill]{parskip}    		% Activate to begin paragraphs with an empty line rather than an indent
\usepackage{graphicx}				% Use pdf, png, jpg, or eps§ with pdflatex; use eps in DVI mode
								% TeX will automatically convert eps --> pdf in pdflatex		
\usepackage{amssymb}

%SetFonts

%SetFonts


\title{625.603 - Statistical Methods and Data Analysis:
Module 1 Assignment 1}
\author{Cameron McIntyre}
%\date{}							% Activate to display a given date or no date

\begin{document}
\maketitle

\section{2.2}
\subsection{ Question 2.2.4}

2.2.4. Suppose that two cards are dealt from a standard 52-card poker deck. Let A be the event that the sum of the two cards is 8 (assume that aces have a numerical value of 1). How many outcomes are in A?

Answer:
We use the counting Principle:
\begin{equation}
    \label{simple_equation}
    \alpha = \sqrt{ \beta }
\end{equation}
%\begin{equation}
%{{6}\choose{1}}{4}\choose{1} +{4}\choose{1}{3}\choose{1}
%\end{equation}
%\subsection{ Question 2.2.11}
%\subsection{ Question 2.2.28}
%\subsection{ Question 2.2.40}
\section{2.2}
%\subsection{ Question 2.3.2}
%\subsection{ Question 2.3.12}
%\subsection{ Question 2.3.16}
\section{2.4}
%\subsection{ Question 2.4.7}
%\subsection{ Question 2.4.10}
%\subsection{ Question 2.4.16}
%\subsection{ Question 2.4.46}
\section{2.5}
%\subsection{ Question 2.5.5}
%\subsection{ Question 2.5.16}
%\subsection{ Question 2.5.32}
%\subsection{ Monte Carlo Exercise}


Text: Problems 2.2.4, 2.2.11, 2.2.28, 2.2.40, 2.3.2, 2.3.12, 2.3.16, 2.4.7, 2.4.10, 2.4.16, 2.4.23, 2.4.46, 2.5.5, 2.5.16, 2.5.32, +Monte Carlo Exercise. All problems should be turned in, including those worked in the collaborative discussion space.

Monte Carlo exercise: John has integers 1:10. He randomly draws 5 without replacement and reasons that he could estimate the 80th percentile of his 10 integers, the value 8, by taking the 2nd largest sampled value; that is the 4th value in order from smallest to largest. (a) Applying this approach repetitively, what proportion of the time will he accurately estimate the value 8? (b) Underestimate? (c) Overestimate? The answer is easily accessible using combinations in the next module, but until then, simulation is the preferred approach.

This assignment is due by Day 7 of Module 1.

\end{document}  